\section{Test Generation, Running and Plotting}

\subsection{The Basic concept}
The test-system uses the classes \texttt{TestRun} and \texttt{TestSeries}, which can be found in package \textit{test}, as the basis for handling the data.
A \texttt{TestRun} represents one test with the parameters, results and the complete \texttt{Scenario}.
The \texttt{TestSeries} consists of at least one \texttt{TestRun}, a name and the "TestGenerator" which created the tests.

The test-system consists of two parts.
The first part includes classes which are generating tests with parameters or working with them.
This category contains the \texttt{TestRun}, \texttt{TestSeries}, \texttt{XMLImporter}, \texttt{XMLExporter}, \texttt{PlainFileExporter}, 
\texttt{Plotter} and the "TestGenerators". 

A "TestGenerator" which is an implementation of \texttt{AbstractTestGenerator}, creates the \texttt{TestSeries} with all necessary \texttt{TestRun}s. 
The implementation sets the class of the "TestRunner" which will run the Tests, name of the test-series, class of the generator and the number 
runs for one distinct parameter set.
For use in parameters the types \texttt{Double} and \texttt{String} are allowed.
The followig arrays for are already predefined in \texttt{AbstractTestGenerator}:
\begin{lstlisting}
public static final Double[] numSNodesArray = {100d};
public static final Double[] numVNetsArray = {5d};
protected static final Double[] numVNodesPerVNetArray = {10d};
public static final Double[] alphaArray = {0.5};
public static final Double[] betaArray = {0.5};
\end{lstlisting}
If it's necessary, it's possible to redefine them.


The second part contains classes which execute the tests.
A "TestRunner" runs the different \texttt{TestRun}(s) contained in the given \texttt{TestSeries}.
The implementation of a "TestRunner" sets the "SeedGenerator", "NetworkGenerator" and "ResourceGenerator(s)" and "DemandGenerator(s)".
All these generators may need to contain different parameters, if a \texttt{TestRun} is lacking a necessary parameter an error is thrown.
The "SeedGenerator" is optional, but some other generators rely on it, which leads to an error.
The "TestRunner" may run tests in a given number of parallel running threads, this may lead to wrong results, because some
algorithms are not thread-safe and using static data structures.


\textbf{UML Missing here}




\subsection{Running Tests}
To run a test, it's necessary to create a \texttt{TestSeries}. 
This can be done by using a "TestGenerator" or by importing an existing \texttt{TestSeries} using the \texttt{XMLImporter}.

When using a "TestGenerator" it's necessary to create an implementation of the \texttt{Abstract\-Test\-Generator}.
In the constructor there is a need to call the super-constructor and initialize the list of parameters \textsl{mParams}.
Each entry contains a name and an array of values, which can have \texttt{Double} or \texttt{String} as data type.
The name may contain a "marker", in this case "S1" (see exmplate below). 
If a name contains a marker these parameter will be synchronized which means they will be rotated together in one step.

\begin{lstlisting}
public class ExampleTestGenerator extends AbstractTestGenerator {
	protected static final Double[] numVNodesPerVNetArray = { 5d, 10d };
	public static final Double[] numVNetsArray = { 20d, 4d };
	public static final Double[] kShortestPath = { 3d };
	public static final Double[] cpu_min = { 10d, 100d };
	public static final Double[] cpu_max = { 100d, 200d };
	
	public ExampleTestGenerator(String seriesName) {
		super(ExampleTestRunner.class, seriesName, "tests.scenarios.example.ExampleTestGenerator", 30);
		
		mParams = new ArrayList<SimpleEntry<String, Object[]>>();
		//S1 is a marker, which lets these parameters treated in one round
		mParams.add(new SimpleEntry<String, Object[]>("Waxman_alpha", alphaArray));
		mParams.add(new SimpleEntry<String, Object[]>("Waxman_beta", betaArray));
		mParams.add(new SimpleEntry<String, Object[]>("SNetSize", numSNodesArray));
		mParams.add(new SimpleEntry<String, Object[]>("NumVNodesPerNet", numVNodesPerVNetArray));
		mParams.add(new SimpleEntry<String, Object[]>("NumVNets", numVNetsArray));
		//S1 is a marker, which lets these parameters treated in one round: (min,max) := (10,100), (100,200)
		mParams.add(new SimpleEntry<String, Object[]>("S1:Min_CPU", cpu_min)); //For CPU generators
		mParams.add(new SimpleEntry<String, Object[]>("S1:Max_CPU", cpu_max)); //For CPU generators
	}
}
\end{lstlisting}
Now it is possible to create an object of the \texttt{ExampleTestGenerator} and generate the \texttt{TestSeries} calling the method
\textsl{generateTests()}.
In the example 240 \texttt{TestRun} objects are created, for any possible combination of parameters, 8 in this case with 30 distinct runs for each.

The next step is to run the tests, it is necessary to implement a "TestRunner" using the \texttt{AbstractTestRunner}.
\begin{lstlisting}
public class ExampleTestRunner extends AbstractTestRunner {
	public ExampleTestRunner(XMLExporter exporter) {
		super(new StandardSeedGenerator(), new FixedWaxmanNetworkGenerator(), true, exporter);
		
		//Set generator for resources
		mResGens.add(new FixedCpuResourceGenerator()); //needs parameters: Min_CPU and Max_CPU
		mResGens.add(new IdResourceGenerator());
		
		//Set generator for demands
		mDemGens.add(new FixedCpuDemandGenerator()); //needs parameters: Min_CPU and Max_CPU
		
		//Set Metrics
		mMetrics.add(new AcceptedVnrRatio());
		mMetrics.add(new RejectedNetworksNumber());
	}

	@Override
	protected IAlgorithm prepareRunnerStage2(TestRun tr) {
		NetworkStack ns = tr.getScenario().getNetworkStack();
		IAlgorithm algo = new SubgraphIsomorphismStackAlgorithm(ns,
				new AdvancedSubgraphIsomorphismAlgorithm(false));
		return algo;
	}
}
\end{lstlisting}

The super constructor will be called with the following parameters:
\begin{enumerate}
	\item "SeedGenerator" based on \texttt{AbstractSeedGenerator}, can be null.
	\item "NetworkGenerator"  based on \texttt{AbstractNetworkGenerator}
	\item \texttt{boolean} which says whether the links should be bidirectional or not
	\item \texttt{XMLExporter} which will export the results
\end{enumerate}
Then the generators for "Resources", "Demands" and the metrics need to be set.
At last it's necessary to implement the method \textsl{prepareRunnerStage2()} which creates and prepares the algorithm which is used for the tests.

The last step is to run the tests, done by a simple main-class
\begin{lstlisting}
public class ExampleMain {

	public static void main(String[] args) {
		//Create the TestGenerator and generate the TestSeries
		ExampleTestGenerator gen = new ExampleTestGenerator("Example TestSeries");
		TestSeries series = gen.generateTests();
		
		//Create the XMLExporter which is used, which doesn't export the Networks
		XMLExporter exporter = new XMLExporter("Result.xml", series.getTestSeriesName(), 
		series.getTestGenerator(), false);
		
		//Create the TestRunner and run the tests in 5 parallel threads
		GeneratorTestRunner run = new GeneratorTestRunner(exporter);
		run.runAllTest(series.getAllTestRuns(), 5);
	}
}
\end{lstlisting}




\subsection{Plotting}
The class \texttt{Plotter} in package \textit{plot} provides a simple method to filter the data and create plots in 2D and 3D.
It's possible to filter the data using \textsl{applyFilter*} methods.
Each filter method need to have the following parameters:
\begin{enumerate}
	\item Name of the parameter or metric as String
	\item (Start) Value of the parameter with type \texttt{Double} or \texttt{String} (String only for \textsl{applyFilterEqual} and \textsl{applyFilterNotEqual}
	\item End Value of the parameter with type \texttt{Double} (Only nneded in range filters)
\end{enumerate}
If the statement is true, the test will removed from the list.

The output-methods always need to have a directory name, in which the results will be plotted and the files are named after the metric which is plotted.
A 2D plot \textsl{output2d} need to have one parameter and a 3D plot \textsl{output3d} need to have two parameters.
The ordering is based on the given parameter(s).
\\
\textbf{Be aware:} At the moment it isn't possible to use \texttt{String} based parameters.

\begin{lstlisting}
public class ExamplePlotMain {
	public static void main(String[] args) {
		TestSeries series2 = XMLImporter.importResults("Result.xml");
		
		Plotter p = new Plotter(series2);
		p.applyFilterEquals("AcceptedVnrRatio", 0.0);
		
		p.output2D("ExamplePlot2d", "NumVNodesPerNet", "AcceptedVnrRatio");
		p.output3D("ExamplePlot3d", "NumVNodesPerNet", "Min_CPU", "AcceptedVnrRatio");
	}
}
\end{lstlisting}



\subsection{Writing a Generator}
To run tests there are different types of generators:
\begin{itemize}
	\item "SeedGenerators" based on \texttt{AbstractSeedGenerator} which are creating seeds for random generators
	\item "NetworkGenerators" based on \texttt{AbstarctNetworkGenerator} which are creating the network stack
	\item "ResourceGenerators" based on \texttt{AbstarctResourceGenerator} which are creating a resource
	\item "DemandGenerators" based on \texttt{AbstarctResourceGenerator} which are creating a demand
\end{itemize}


A generator has always implement the methods \textsl{generate} which contains the logic and \textsl{reset} 
which resets the state if needed and is called before a every new \texttt{TestRun}.
If the generator needs parameters, these are specified in the \texttt{GeneratorParameter} annotation.
The parameter array will filled with these values in order of annotation.
The following values are possible:
\begin{itemize}
	\item "Seed:Seed": Seed from "SeedGenerator" from type \texttt{Double}
	\item "Networks:Networks": The \texttt{NetworkStack}
	\item "TR:ParameterName": The value for the parameter with the name "ParameterName" from \texttt{TestRun} (TR)
	\item "Result:package.ResoureOrDemandGenerator": The result of the \textsl{generate} method of the given and used demand or resource generator
	\item "Method:package.ResoureOrDemandGenerator|getFoo": The result of the non-static \textsl{getFoo} method of the given and in this "TestRunner"' used demand or resource generator
	\item "SMethod:some.package.Class|getBar": The result of the static \textsl{getBar} method of the given class "some.package.Class"
 \end{itemize}

If a parameter is not available or there is an error, an \texttt{Error} is thrown.
\textbf{Be aware:}Due to parallel execution of threads the call of static methods may lead to wrong results!


\begin{lstlisting}
@GeneratorParameter(
		parameters = { "Networks:Networks", "TR:Min_CPU", "TR:Max_CPU", "Seed:Seed"}
		)
public class FixedCpuResourceGenerator extends AbstractResourceGenerator<List<CpuResource>> {

	@Override
	public List<CpuResource> generate(ArrayList<Object> parameters) {
		ArrayList<CpuResource> resList = new ArrayList<CpuResource>();
		
		NetworkStack ns = (NetworkStack)parameters.get(0);
		Integer minCPU = ConversionHelper.paramObjectToInteger(parameters.get(1));
		Integer maxCPU = ConversionHelper.paramObjectToInteger(parameters.get(2));
		Long seed = ConversionHelper.paramObjectToLong(parameters.get(3));
		
		Random random = new Random();
		random.setSeed(seed);
		
		SubstrateNetwork sn = ns.getSubstrate();
		
		for(SubstrateNode n : sn.getVertices()) {
			CpuResource cpu = new CpuResource(n);
			int value = (int) (minCPU + (maxCPU
					- minCPU + 1)
					* random.nextDouble());
			cpu.setCycles((double) value);
			n.add(cpu);
			resList.add(cpu);
		}
		return resList;
	}

	@Override
	public void reset() {}
}
\end{lstlisting}
