\section{Adding of New Pair of Resource/Demand Types}

\subsection{Overview}
\begin{itemize}
	\item All resource types are derived from \texttt{AbstractResource}
	\item Resources may only be added to the entities of the \texttt{SubstrateNetwork}
	\item The \texttt{SubstrateNetwork} consists of \texttt{SubstrateLinks} and \texttt{SubstrateNodes}
	\item All demand types are derived from \texttt{AbstractDemand}
	\item Demands may only be added to the entities of a \texttt{VirtualNetwork}
	\item A \texttt{VirtualNetwork} consists of \texttt{VirtualLinks} and \texttt{VirtualNodes}
\end{itemize}
We use the \textbf{Visitor pattern}\footnote{See \href{http://en.wikipedia.org/wiki/Visitor_pattern}{http://en.wikipedia.org/wiki/Visitor\_pattern}} 
and \textbf{Adapter pattern}\footnote{See \href{http://en.wikipedia.org/wiki/Adapter_pattern}{http://en.wikipedia.org/wiki/Adapter\_pattern}}  to avoid
\begin{itemize}
	\item casts to concrete demand/resource classes,
	\item the use of Java's instanceof as far as possible.
\end{itemize}
\textbf{N.b.: } We use these design pattern in an entangled way.
\begin{itemize}
	\item A resource is a visitor for a demand providing the occupy and free visitors
	\item A demand is a visitor for a resource providing the accepts and fulfills visitors
	\item Do not get confused about this.
	\item Every Resource and Demand has a name. When you create a clone (with \textsl{getCopy()}) you have to copy the name over to the clone. 
	So that the equals()-Method can do matches with clones.
	\item If you don't provide a name, the name is set to \textbf{NameOfOwner\_this.hashCode()}
\end{itemize}


\subsection{About the visitor pattern}
Normally the visitor pattern is used to be able to add new operations to an existing object structure, 
without altering every single object. 
The operations get encapsulated in visitors, 
which then visit the objects and interacts with them. 
The object only have to provide a simple interface to accept visitors.
Here we already know all operations we need (free, occupy, accepts, fulfills), 
but want to be able to add new resource / demand pairs to our network, e.g. new objects, 
which then can handle all upcoming resource / demand requests.

%TODO: Need to add the Constructor Parameters
\subsection{Constraints}
For ALEVIN's \textbf{XML exchange format}\footnote{See SVN\_base/src/XML/Alevin.xsd}, resource and demand classes need to meet the following constraints
\begin{itemize}
	\item The constraint classes must implement one or both of the \texttt{INodeConstraint}, \texttt{ILinkConstraint} interfaces. This shows whether the Constraint is applicable to nodes, links or both.
	\item For each parameter that shall be included in the exchange format, a getter and a setter method \textbf{must be declared and  annotated} with @ExchangeParameter 
	\begin{itemize}
		\item Getters are required for export
		\item Setters are required for import
	\end{itemize}
	\item The parameters for these methods \textbf{must not} be simple types (int, double) but classes 
	such that it can be used via Java Reflection
	\begin{itemize}
		\item N.B.  Currently, the exchange format supports the following types: Integer, Double, String, Boolean and ArrayList<String>
	\end{itemize}
	\item Resources
	\begin{itemize}
		\item Getter methods must be named according to the following pattern: get + <parameter name>
		\item Setter methods must be named according to the following pattern: set + <parameter name>
	\end{itemize}
	\item Demands
	\begin{itemize}
		\item Getter methods must be named according to the following pattern: getDemanded + <parameter name>
		\item Setter methods must be named according to the following pattern: setDemanded + <parameter name>
	\end{itemize}
	\item Constructors
	\begin{itemize}
		\item It's suggested that you use a default constructor only with the owner as parameter.
		\item If you need to have additional parameter you need to add the @ConstructionParameter annotation
		\item If you have more than one parameter, the parameterName and parameterGetter have to be in the right order.
		\begin{lstlisting}
@AdditionalConstructParameter(
		parameterNames = {"sNetwork"},
		parameterGetters = {"getsNetwork"}
)
public final class IdResource extends AbstractResource implements
		INodeConstraint {
	private String id;
	private final SubstrateNetwork sNetwork;


	public IdResource(Node<? extends AbstractConstraint> owner, SubstrateNetwork sNetwork) {
		super(owner);
		this.sNetwork = sNetwork;
	}
	...
		\end{lstlisting}
	\end{itemize}
\end{itemize}


\subsection{Procedure}
To add the new demand type \texttt{MyNewDemand} and the corresponding resource type \texttt{MyNewResource} perform the following steps: 
\begin{enumerate}
	\item Add dummy methods to
	\begin{itemize}
		\item \textsl{vnreal.demands.DemandVisitorAdapter }
		\begin{lstlisting}
public boolean visit(MyNewDemand req) {
  return false;
}
		\end{lstlisting}
		\item \textsl{vnreal.resources.ResourceVisitorAdapter }
			\begin{lstlisting}
public boolean visit(MyNewResource req) {
  return false;
}
		\end{lstlisting}
	\end{itemize}
	\item Create a new class \texttt{MyNewResource} in package \textit{vnreal.resources} extending \texttt{AbstractResource}
	\item Create a new class \texttt{MyNewDemand} in package \textit{vnreal.demands} extending \texttt{AbstractDemand}
	\item Implement all abstract methods (imitate existing code, like \texttt{IdDemand} and \texttt{IdResource}) 
\end{enumerate}
